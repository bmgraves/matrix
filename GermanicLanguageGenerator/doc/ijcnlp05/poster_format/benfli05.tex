%
% File ijcnlp-05.tex
% February 2005

\documentclass[11pt]{article}
\usepackage{ijcnlp-05}
\usepackage{times}
\usepackage{latexsym}

\newcommand{\hpsg}{\textsc{hpsg}}
\newcommand{\lkb}{\textsc{lkb}}
\newcommand{\lfg}{\textsc{lfg}}



\title{Rapid Prototyping of Scalable Grammars: Towards Modularity in Extensions to a Language-Independent Core}

\author{Emily M.~Bender\\
  Department of Linguistics\\
  University of Washington\\
  Box 354340\\
  Seattle WA 98195-4340 USA\\
  {\tt ebender@u.washington.edu} \And
  Dan Flickinger \\
  Center for the Study of Language and Information\\
  Stanford University\\
  Stanford CA 94305-2150 USA\\
  {\tt danf@csli.stanford.edu}}

\date{}

\begin{document}

\maketitle

\begin{abstract}
We present a new way to simplify the construction
of precise broad-coverage grammars, employing typologically-motivated,
customizable extensions to a language-independent core
grammar.  Each `module' represents a salient dimension of
cross-linguistic variation,
and
presents the grammar developer with simple choices that result in
automatically generated language-specific software.  We
illustrate the approach for several phenomena
and explore the interdependence of the modules.
\end{abstract}

\section{Introduction}

Manual development of precise broad-coverage grammar implementations,
useful in a range of natural language processing/understanding tasks,
is a labor-intensive undertaking, requiring many years of work by
highly trained linguists.  Many recent efforts toward reducing the
time and level of expertise needed to produce a new grammar have
focused on adapting an existing grammar of another language
\cite{Butt-et-al-02,Kim:Dal:Kap:Kin:Mas:Ohk:03,Bat:Kru:Kru:ta}.  Our
work on the `Grammar Matrix' has pursued an alternative approach,
identifying a set of language-independent grammar constraints to which
language-specific constraints can be added~\cite{Ben:Fli:Oe:02}.  This
approach has the hitherto unexploited potential to benefit from the
substantial theoretical work on language typology.
%, which characterizes
%linguistic phenomena and the varied mechanisms that languages employ
%to realize them.  
In this paper, we present a prototype Grammar Matrix
customization system.  This system draws on phenomenon-specific
modules encoding dimensions of linguistic variation, presents the
grammar developer with simple choices for each phenomenon, and then
automatically generates a working starter-grammar, incorporating both
the cross-linguistic Matrix core and language-specific constraints.
The prototype addresses basic word order, sentential
negation, yes-no questions, and a small range of lexical entries.

\section{The Grammar Matrix}

Wide-coverage grammars representing deep linguistic analysis exist in
several frameworks, including Head-Driven Phrase Structure Grammar
(\hpsg), Lexical-Functional Grammar, and Lexicalized Tree Adjoining
Grammar.  In \hpsg\ \cite{Pol:Sag:94}, the most extensive grammars are
those of English \cite{Flickinger:00}, German
\cite{Hinrichs:etal:97,Mue:Kap:00,Crysmannip}, and Japanese
\cite{Siegel:00,Siegel:Bender:02}.  The Grammar Matrix is an attempt
to distill the wisdom of existing grammars and document it in a form
that can be used as the basis for new grammars. The main goals of the
project are: (i) to develop in detail semantic representations and 
the syntax-semantics interface, consistent with other work
in \hpsg; (ii) to represent generalizations across linguistic objects
and across languages; and (iii) to allow for very quick start-up as
the Matrix is applied to new languages.
%, using the inventory of types
%and the built-in links to the \lkb\ grammar engineering environment
%\cite{Copestake:02} and the {\sc pet} system \cite{Callmeier:00}.

The original Grammar Matrix consisted of types defining the basic
feature geometry, types associated with Minimal Recursion Semantics
(e.g., \cite{Cop:Las:Fli:01}), types for lexical and syntactic rules,
and configuration files for the \lkb\ grammar development environment
\cite{Copestake:02} and the {\sc pet} system \cite{Callmeier:00}.
Subsequent releases have refined the original types and developed a
lexical hierarchy.  The constraints in this `core' Matrix are intended
to be language-independent and monotonically extensible in any given
grammar.  With the typology-based modules presented here, we extend
the constraint definitions which can be supplied to grammar developers
to those that capture generalizations holding only for subsets of
languages.

\section{Typology-based modules}

In general, we find two kinds of typological variation across
languages.  On the one hand, there are systems (formal or functional)
which must be represented in every language.  For example, every
language has some set of permissible word orders (formal) and a means
of expressing sentential negation (functional).  On the other hand,
there are linguistic phenomena which appear in only some languages,
and are not typically conceptualized as alternative realizations of
some universal function, phenomena such as noun
incorporation, numeral classifiers, and auxiliary verbs. Each of these
phenomena are found in recurring varieties that can be subjected to
typological analysis (see, e.g., \cite{Mithun:84}).  Our approach is
designed to handle both kinds of typological variation.

As with earlier versions of the Matrix, we aim to support rapid
prototyping of precision grammars that can scale up to broad-coverage
(as have the NorSource \cite{Hel:Hau:03} and Modern Greek
\cite{Kor:Neu:2003} grammars, based on early versions of the
Matrix). This sets a high bar for the modules themselves, requiring
them to be good early approximations which may need to be refined but
not thrown out.  It also requires that the automatically generated
grammar files maintain a high degree of readability so that they may
be effectively modified.  In future work, we intend 
to extend the system to allow the linguist to revise
decisions in the face of new information or improved linguistic
analyses.

The core Matrix and modular extensions to it may appear analogous to
the Principles and Parameters proposed by \newcite{Chomsky81} and
others.  However, whereas Parameters are meant to be abstract
`switches' which simultaneously control multiple different, apparently
unrelated phenomena, the modules in the Matrix each encode the
constraints necessary to handle one particular phenomenon.
Nonetheless, this does not make the modules trivial: they need to be
carefully designed in order to be mutually consistent, ideally across
all possible combinations.  Our strategy is thus consistent with a
bottom-up, data-driven investigation of linguistic universals and
constraints on cross-linguistic variation.  As the number and breadth
of implemented grammars grows, we expect linguistic predictions to
emerge and become part of improved modules, particularly with respect
to interactions among the distinct phenomena covered.  Our approach
should in time be instrumental in assisting large-scale typological
investigations (covering hundreds of languages), making use of the
linguistically precise constraints encoded in these modules to uncover
deeper and more subtle facts about languages.

\section{Implementations of prototype system}
\label{imp}

%To explore the issues that will arise in developing
%mutually consistent modules,
We have implemented a prototype system with a small set of modules
targeting basic word order, main-clause yes-no questions, and
sentential negation.\footnote{\newcite{Dre:Ben:05} present a module
for coordination which is integrated with those described here.}  The
corresponding choices and a questionnaire for creating a small lexicon
are presented to the user through an html form interface. A perl/cgi
back-end produces a starter grammar from the user input and an
internal knowledge base.  The resulting grammars can be used
immediately to parse and generate a fragment of the target language.
The system can be accessed at
http://www.delph-in.net/matrix/modules.html.  This section describes
its linguistic coverage.

\subsection{Word order}

The word order module addresses the so-called basic word order in a
language: the relative order of subjects, verbs, and verbal
complements.  Languages vary in their rigidity in this respect, 
and the question of how to determine the basic word-order of a
language is notoriously complex.  Nonetheless, we believe that most
linguists working on linguistic description analyze some orders
as primary and others as derived.  Thus the word order module is
meant to capture the relative ordering
of major constituents in clauses without
word-order changing phenomena such as topicalization,
extraposition, subject-verb inversion, etc.
Modules for such phenomena will need to interact
appropriately with the basic word-order module.

The Matrix core grammar provides definitions of basic head-complement
and head-subject schemata which are consistent with our
implementation of compositional semantics \cite{Fli:Ben:03}, as well
as definitions of head-initial and head-final phrase types.  
%These
%rules are also consistent with the {\hpsg}-style handling of
%long-distance dependencies implemented in the Matrix.
%Cut this footnote for space reasons
%\footnote{We note that the long-distance dependency machinery
%might be best treated as a module itself rather than part of the core
%grammar.}  
The word order module
creates subtypes joining the head-complement and
head-subject schemata with the types specifying head/dependent
order, creates instances of those types as required by
the \lkb\ parser, and constrains the rules to eliminate spurious ambiguity in the case of
free word order.  It currently handles SOV, SVO,
VSO, VOS, OVS, OSV, V-final, V-initial, and free word order.  We leave
to future work variations such as V2 order, differing word order
in main v.\ subordinate clauses, 
and flexible ordering among
complements in otherwise strict word order languages.  

\subsection{Yes-no questions}

For yes-no questions, we implement four alternatives:
inversion of the subject and a main or auxiliary verb relative
to declarative word order and sentence-initial
or final question particles.

%EB: Comment out the following since we don't really do anything with this
%%; and (iv) use of intonation alone to indicate a question.

Inversion of the subject and the main verb is implemented with a
lexical rule which relocates the subject (the value of {\sc subj} in
the valence specifications) to be the first on the {\sc comps} list,
and further assigns a positive value for an additional feature {\sc
inv} (inverted) on verbs.  This feature may well have independent
syntactic motivation in the language, but is in any case used here so
the declarative/interrogative distinction can be made in the semantics
once the clause is constructed.
Subject-aux inversion is a minor extension of the basic inversion
type, constraining the lexical rule to only apply to auxiliary
verbs.  This module handles `support' verbs like {\it do} in English
in not licensing inversion with main verbs, while
licensing similar strings with a semantically empty support verb
(if it is in the lexicon).  
%There is no mechanism as yet to
%bar such `support' verbs from neutral declarative sentences.
The third type of mechanism employs a distinct question particle, here
treated as a pre- or post-modifying sentence adverb.  The grammar developer is
prompted for this positional distinction, and for the spelling of the
particle; the code for the relevant lexical entry is then
autogenerated, instantiating a question particle type which supplies
the remaining syntactic and semantic constraints needed.

Future work on this module includes support for `intonation questions',
where the same string can be associated with either proposition or question
semantics, as well as the integration of declarative/interrogative punctuation
contrasts.

\subsection{Sentential negation}

The sentential negation module handles two general negation strategies,
several variants on each, and allows for both to coexist in a single grammar.

The first strategy is negation via verbal inflection.  For this
strategy, the grammar developer specifies whether the inflection
attaches to main verbs, auxiliaries, or either; whether it is a prefix
or a suffix; and the form of the affix.  We currently only allow for
strictly concatenative morphology.  In a more fully developed system,
the syntax-semantics modules here would be interfaced with a separate
means of specifying morphophonology (cf.\ \cite{Ben:Goo:05}).

The second strategy is negation via a negative adverb, with two
sub-cases: The negative adverb may be an independent modifier (of V,
VP, or S and pre- or post-head) or it may be a
selected complement of the verb (main verbs only, auxiliaries only, or
both) \cite{Kim00}.  The grammar developer specifies the form of the adverb.

Neither, either or both of these strategies may be selected.
If neither, the grammar
produced will not contain an analysis of negation.  If both,
the grammar developer must specify how the strategies
interact, from among five choices:
(i) the two strategies are in complementary distribution, (ii) the two
strategies can appear independently or together, (iii) both inflection
and an adverb are required to express sentential negation, (iv) the
adverb is obligatory, but it may appear with or without the
inflection, and (v) the inflection is obligatory, but it may appear
with or without the adverb.

In the generated grammars, independent adverbs are implemented by
adding appropriate lexical types and lexical entries.  Selected
adverbs and inflection are handled via lexical rules similar to those
presented in \cite{Sag:Was:Ben:03}.  For example, in a language where
sentential negation can be expressed by inflection alone or inflection
in combination with a (selected) adverb, we generate two lexical
rules.  One changes the form of the verb and adds the negative
semantics.  The other changes the form of the verb and adds the
negative adverb to its complements list.

\subsection{Lexicon}

As \hpsg\ is a strongly lexicalist theory, words
tend to carry quite a bit of information.  This information is encoded
in lexical types; lexical entries merely specify
the type they instantiate, their orthographic form, and their semantic
predicate.  Many of the constraints required (e.g., for the linking of
syntactic to semantic arguments) are already provided by the core
Matrix.  However, there is also cross-linguistic variation.

We ask the grammar developer to specify two nouns and two verbs (one
transitive and one intransitive), as well as an auxiliary, two
determiners, two case-marking adpositions, a negative adverb and a
question particle, if appropriate.  Nouns are specified as to whether
they require, allow, or disallow determiners.  Verbs are specified as
to whether each argument is expressed as an NP or a PP, and optionally
for an additional (non-finite) form.  Auxiliaries are specified as
to whether they introduce independent predicates or only carry
tense/aspect; take S, VP or V complements; appear to the left, right
or either side of their complements; and take NP or PP subjects.
Case-marking adpositions must be specified as either prepositions or
postpositions.  Finally, the questionnaire requires orthographic
forms and predicate names.  Note that the forms are assumed to be
fully inflected (modulo negation), support
morphological processes awaiting future work.

We use this information and the knowledge
base to produce a set of lexical types inheriting from the types
defined in the core Matrix and specifying appropriate
language-specific constraints, and a set of lexical entries.

%% \subsection{Summary}

%% This section has briefly described the coverage of our existing
%% modules and customization interface, which may be accessed at:
%% http://www.delph-in.net/matrix/modules.html. We now turn to 
%% what the development of this system has taught us about the
%% interdependence of modules.

\section{Limits of modularity}

Recent computational work in \hpsg\ has asked whether
different parts of a single grammar can be abstracted
into separate, independent modules, either for processing 
%(parsing, generation) 
\cite{Kasper:Krieger:96,Theofilidis:etal:97} or grammar
development \cite{Keselj:01}.  Our work is most similar to Ke{\v{s}}elj's
though we are pursuing different goals: Ke{\v{s}}elj\nocite{Keselj:01}
is looking to support a division of labor among multiple
individuals working on the same grammar and to support variants of a
single grammar for different domains.  
%Ke{\v{s}}elj\nocite{Keselj:01} models his modules on object-oriented
%programming, allowing 
His modules each have private and public features and types, and he
illustrates the approach with a small-scale question answering system.
In contrast, we are approaching
this issue from the perspective of reuse of grammar code in the
context of multilingual grammar engineering (a possibility suggested,
but not developed, by Theofilidis et al).

Our notion of modularity is influenced by the following constraints:
(i) The questions in the customization interface must be sensible to
the working linguist; (ii) The resulting starter grammars must be
highly readable so that they can be extended by the grammar developer
(typically only one per grammar); and (iii) \hpsg\ practice values capturing linguistic generalizations by having single
types encode many different constraints and, ideally, single
constraints contribute to the analysis of many different phenomena.
% and (iv) The
%modules themselves must be reasonably easy to maintain, which
%precludes precompiling solutions for each module as it would be used
%in the context of relevant choices in other dimensions.


Even with the modest linguistic coverage of the existing
system, we have found many cases of non-trivial interaction between
the modules: Our phrase structure rules, following \hpsg\ practice,
capture cross-categorial generalizations: if both verbs and
adpositions follow their complements, then a single complement-head
rule serves for both.  However, few languages (if any) are completely
consistent in their ordering of heads and dependents.  Thus, before
defining the types and instances for these rules, we must determine
whether the fragment requires auxiliaries (for negation or yes-no
questions) or case-marking adpositions, and whether their order with
respect to their complements is consistent with that of main verbs.  A
second example is the lexical type for main verbs, whose definition
depends on whether the language has auxiliaries (requiring a feature
{\sc aux} distinguishing the two kinds of verbs and a feature {\sc
form} governing the distribution of finite and non-finite verbs).
%This situation will only get
%more complex as we add modules, e.g., for case and agreement. 
As a third example, the negation and question modules each have
options requiring auxiliaries, but we must posit the associated types
and constraints at most once.

Thus we find that, for our purposes, the relevant notion of modularity
is modularity from the point of view of the linguist who uses the
system to create a starter grammar.  To support this, we strive to
make the questions we ask of the linguist be as independent of each
other as possible, and to make it clear when one particular choice
(e.g., negation as inflection) requires further information (suffix
v.\ prefix).  The fact that the questions we present to
the linguist don't correspond to neatly isolated parts of the
underlying knowledge base is not a failure of the approach, but
rather a reflection of the complexity of language.
% de Saussure's
%{\it syst\`eme o\`u tout se tient}.  
The very interconnectedness of
grammatical phenomena is at the heart of research in theoretical
syntax.  We intend our system to provide a data-driven
cross-linguistic exploration of that interconnection.

\section{Validation of prototype system}


To verify the mutual consistency of the modules developed so far 
and to illustrate their applicability to a interesting range of
languages, we developed abstract test suites for seven languages.
This convenience sample of languages is not representative, either
typologically or genetically.  The grammatical and ungrammatical
examples in each test suite use a small, artificial lexicon, and
reflect the typological properties of each language along the
dimensions of basic word order, sentential negation,
and yes-no questions
(Table~\ref{testsuitetable}). 
Table~\ref{testresults} presents the performance
of each grammar (as generated by our prototype system with appropriate
input) on its associated test suite.

\begin{table}[ht]
\begin{center}
\small
\begin{tabular}{llll}
\hline
Language\footnotemark & Order & Negation & Yes-no Q\footnotemark \\ \hline
English  & SVO	      & aux-selected adv& aux inv \\
Hindi    & SOV        & pre-V adv 	& S-init part.\\
Japanese & V-final    & verbal suffix   & S-final part\\
Mandarin & SVO	      & pre-V adv	& S-final part,\\
         &            &                 &  A-not-A \\
Polish   & free	      & pre-V adv       & S-init part  \\
Slave    & SOV        & post-V adv	& S-init part \\
Spanish  & SVO	      & pre-V adv	& main V inv \\ \hline
\end{tabular}\vspace{-10pt}
\end{center}
\caption{Languages used in testing}
\label{testsuitetable}
\end{table}

\addtocounter{footnote}{-1}
\footnotetext{Sources: Hindi: Snell and Weightman, 2000\nocite{Sne:Wei:00},
Mandarin: Li and Thompson, 1981\nocite{Li*81}, Polish: Adam Przepi\'{o}rkowski, p.c.,
Slave (Athabaskan): Rice, 1989\nocite{Rice89}}
\addtocounter{footnote}{1}
\footnotetext{In addition to intonation questions, if any.}

\begin{table}[ht]
\begin{center}
\small
\begin{tabular}{lrrrr}
\hline
Language & Pos. & Coverage & Neg. & Overgen. \\ \hline
English  & 5	     & 100\%	& 10 & 10\%\\
Hindi    & 5         & 100\% 	& 10 & 0\% \\
Japanese & 6         & 100\%    & 10 & 0\% \\
Mandarin & 4	     &  75\%	&  9 & 0\% \\
Polish   & 14	     & 100\%    &  8 & 0\% \\
Slave    & 3         & 100\%	&  6 & 0\% \\
Spanish  & 5         & 100\%	&  7 & 0\% \\ \hline
\end{tabular}\vspace{-10pt}
\end{center}
\caption{Parsing evaluation results}
\label{testresults}
\end{table}

While these test suites are quite modest, we believe they show that
the prototype system is able to produce good first-pass grammar
fragments for an interesting variety of languages.  More study is
needed to develop a means of testing the
cross-compatibility of all choices on all modules, to evaluate
the coverage against a typologically justified sample,
and to gauge the success of this strategy in producing 
grammars which are comprehensible to beginning grammar developers.

\section{Conclusion and outlook}

We have described a method for extending a language-independent core
grammar like the Grammar Matrix with modules handling
cross-linguistically variable but still recurring patterns.  This
method allows for extremely rapid prototyping of deep precision
grammars in such a way that the prototypes themselves can serve as the
basis for sustained development.  We envision at least four potential
uses for this kind of grammar prototyping: (i) in pedagogical
contexts, where it would allow grammar engineering students to more
quickly work on cutting-edge problems, (ii) in language documentation,
where a documentary linguist in the field might be collaborating
remotely with a grammar engineer to propose and test hypotheses, (iii)
in leveraging the results from economically powerful languages to
reduce the cost of creating resources for minority languages, 
%(iv) in
%presenting typological properties in machine-readable format to
%improve ML algorithms which might need to be tuned differently for
%different kinds of languages, and 
and (iv) in supporting typological or
comparative studies of linguistic phenomena or interactions between
phenomena across languages.

\section*{Acknowledgments}

We thank Scott Drellishak, Stephan Oepen, Laurie Poulson, and the 2004
and 2005 multilingual grammar engineering classes at the University of
Washington for valuable input and NTT Communication Science
Laboratories for their support through a grant to CSLI (Stanford). All
remaining errors are our own.

\bibliographystyle{ijc}
%% TODO: use base name of your .bib file
\bibliography{modules}     

\end{document}
