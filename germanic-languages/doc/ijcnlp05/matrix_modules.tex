\documentstyle[robbib]{llncs}
%

\newcommand{\hpsg}{\textsc{hpsg}}
\newcommand{\lkb}{\textsc{lkb}}
\newcommand{\lfg}{\textsc{lfg}}


% This is LLNCS.DEM the demonstration file of
% the LaTeX macro package from Springer-Verlag
% for Lecture Notes in Computer Science, version 1.1

\begin{document}

\title{Rapid Prototyping of Scalable Grammars: Towards Modularity in Extensions to a Language-Independent Core}

\author{anonymous}

\institute{anonymous}

\maketitle

\begin{abstract}
This paper presents a novel approach to simplifying the construction
of precise broad-coverage grammars, employing typologically-motivated
modules as customizable extensions to a language-independent core
grammar.  Each module represents some salient dimension of
cross-linguistic variation such as word order or negation, and
presents the grammar developer with simple choices that result in
automatically generated language-specific software that becomes part
of the grammar implementation for that language.  The paper
illustrates this notion of a module for several phenomena, and
includes a validation of the approach testing a prototype against
multilingual test data reflecting interactions among these phenomena.
\end{abstract}
%

\section{Introduction}
Manual development of precise broad-coverage grammar implementations,
useful in a range of NLP and NLU tasks, 
is a labor-intensive undertaking, typically requiring multiple years
of work by highly trained computational linguists.  Recent
efforts toward reducing the time and level of expertise
needed for producing a new grammar have focused either on adapting an
existing grammar of another language
\cite{Butt-et-al-02,Kim:Dal:Kap:Kin:Mas:Ohk:03,Bat:Kru:Kru:ta}, or on
identifying a set of language-independent grammar constraints (a `Grammar
Matrix') to which
language-specific constraints can be added~\cite{Ben:Fli:Oe:02}.
Neither implementation approach has benefited from the substantial
theoretical work on language typology, which characterizes linguistic
phenomena and the varied mechanisms that languages employ to realize
them.  In this paper we report on an elaboration of the Grammar Matrix
approach in which phenomenon-specific modules encode the dimensions of
variation, and present the grammar developer with simple choices 
for each phenomenon. Each module then
automatically generates the necessary language-specific constraints as
extensions to the Matrix grammar files, enabling the linguist to
produce a working prototype grammar with very little effort or
computational expertise.  We illustrate this concept of a
typology-based module for basic word order, sentential negation, and
yes-no questions, and we conclude with a validation of a prototype of
the system by testing the resulting grammar implementations for a variety
of language types.

\section{The Grammar Matrix}

The past decade has seen the development of wide-coverage implemented
grammars representing deep linguistic analysis in
several frameworks, including Head-Driven Phrase Structure Grammar
(\hpsg), Lexical-Functional Grammar (\lfg), and Lexicalized Tree
Adjoining Grammar ({\sc ltag}). In \hpsg\ \cite{Pol:Sag:94}, the most
extensive grammars are those of English \cite{Flickinger:00}, German
\cite{Hinrichs:etal:97,Mue:Kap:00,Crysmannip}, and Japanese \cite{Siegel:00,Siegel:Bender:02}.
The Grammar Matrix \cite{Ben:Fli:Oe:02} is presented as 
an attempt to distill the
wisdom of existing grammars and document it in a form that can
be used as the basis for new grammars. The main goals of the project
are: (i) to develop in detail semantic representations and in particular 
the syntax-semantics interface, consistent with other work in \hpsg; 
(ii) to represent generalizations across linguistic objects
and across languages; and (iii) to allow for very quick start-up as 
the Matrix is applied to new languages, using the inventory of types and
the built-in links to the \lkb\ grammar engineering environment
\cite{Copestake:02} and the {\sc pet} system \cite{Callmeier:00}.

The first, preliminary version of the Grammar Matrix 
consisted of types defining the basic feature geometry and technical
devices (e.g., for list manipulation), types associated with Minimal
Recursion Semantics (see, e.g., \cite{Cop:Las:Fli:01}), types for
lexical and syntactic rules, and configuration and parameter
files for the \lkb\ and {\sc pet}.  The current Matrix release further 
includes a hierarchy of lexical types for creating language-specific 
lexical entries.

All of the constraints in the current Matrix are intended to be 
language-independent, and monotonically extensible when developing a
new grammar for a language.
%, though not all types defined in the Matrix
%will necessarily be employed for any given language.  
This strong 
assumption of universality was of practical utility in the early
development of the Matrix, but it sharply limited the 
%inventory of
constraint definitions that could be supplied to grammar developers
since many generalizations hold only for subsets of languages.  It is
this limitation of the current Matrix that we address by introducing
the notion of typology-based modules.

\section{Typology-based modules}

In general, we find two kinds of typological variation across
languages.  On the one hand, there are systems (formal or functional)
which must be represented in every language.  Word order is a prime
example of a formal system represented in every language: Every
language has some set of permissible word orders.
A functional system which is represented in every language is
sentential negation, expressed using a variety of mechanisms.  On 
the other hand, there are linguistic
phenomena which appear in only some languages, and are difficult to
conceptualize as alternative realizations of some universal function.  More
importantly, for our purposes, ordinary working linguists (`OWLs')
don't conceptualize them in this way.  These include things like
noun incorporation, numeral classifiers, and auxiliary verbs; each of
these phenomena are found in recurring varieties that can be 
subjected to typological
analysis (see, for example, \cite{Mithun:84} on noun incorporation).
The type of modular extension to the Matrix which we propose here
is meant to scale up to handle both kinds of typological variation.

Consistent with earlier versions of the Matrix, we aim to support 
rapid prototyping of precision grammars that can scale up to
broad-coverage (as have the NorSource \cite{Hel:Hau:03} and
Modern Greek \cite{Kor:Neu:2003} grammars, both based on an early
version of the Matrix). This sets a high bar for the modules
themselves, requiring them to be not only quick first-pass solutions,
but also good early approximations which may need to be refined but not
thrown out.  It also requires that the automatically generated grammar
files maintain a high degree of readability so that they may be
effectively modified.  We will also encounter issues with updating
procedures for the core Matrix while maintaining consistency with the
module-derived customizations for individual languages, and issues
with enabling the linguist to go back and revise decisions in the face
of new information or improved linguistic analyses (see \S\ref{management}).

\section{Related work}

The core Matrix and modular extensions to it may appear at first
glance to be analogous to the Principles and Parameters proposed by
\namecite{Chomsky81} and others.  However, they differ in at least one
important respect: Whereas Parameters are meant to be abstract
`switches' which simultaneously control multiple different, and
apparently unrelated, surface phenomena, the modules in the Matrix
each directly encode the constraints necessary to handle one
particular phenomenon.  Nonetheless, this does not make the modules
themselves trivial: they need to be carefully designed in order to be
mutually consistent, ideally across all possible combinations.  Our
strategy is thus consistent with a bottom-up, data-driven
investigation of linguistic universals and constraints on
cross-linguistic variation.  As the number and breadth of implemented
grammars grows, we expect linguistic predictions to emerge and become
part of improved modules, particularly with respect to interactions
among the distinct phenomena that the modules cover.  Our approach
should in time be instrumental in assisting large-scale typological
investigations (covering hundreds of languages), making use of the
linguistically precise constraints encoded in these modules to uncover
deeper and more subtle facts about languages.

Another vein of related work is that focusing on modularization of
\hpsg\ grammars, such as \citeboth{Kasper:Krieger:96},
\citeboth{Theofilidis:etal:97}, and \citeboth{Keselj:01}. Whereas
\citeauthor{Kasper:Krieger:96} and to some extent
\citeauthor{Theofilidis:etal:97} focus on modular {\it processing}
based on integral \hpsg\ grammars, the present work, like
\citeboth{Keselj:01}, attempts to factor the grammar itself into
separate pieces that can be combined in different ways.  Our work
differs from that of \citeauthor{Keselj:01} in two important respects:
(i) \citeauthor{Keselj:01} is looking to modularity to support a
division of labor among multiple individuals working on the same
grammar and to support variants of a single grammar for different
domains.  In contrast, we are approaching this issue from the
perspective of reuse of grammar code in the context of multilingual
grammar engineering (a possibility suggested, but not developed, by
\namecite{Theofilidis:etal:97}). (ii) \citeauthor{Keselj:01} models
his modules on OOP, allowing each private as well as public features
and types.  He illustrates the system an example from a
small-scale question answering system.  While we share \citeauthor{Keselj:01}'s
interest in collaborative grammar engineering, we have found that 
with broad-coverage grammars, the kinds
of linguistic subsystems which are most usefully
split off into modules are, in our experience, tightly woven
together in a working grammar of a language.  To attempt to isolate
them in mutually independent modules goes against the way language,
de Saussure's {\it syst\`eme o\`u tout se tient}, seems to work.

We nonetheless see value in producing mutually consistent
components which can be combined in various ways to model
typologically diverse languages, even without means of 
preventing changes to the core grammar or one module from requiring
changes to other modules.  The very interconnectedness of grammatical
phenomena is at the heart of research in theoretical syntax.  We
intend our system to provide a data-driven cross-linguistic
exploration of that interconnection.

\section{Implementations of three modules}
\label{imp}

To explore the issues that will arise in developing and managing
mutually consistent modules, we have
implemented an initial set of modules targeting basic
word order, main-clause yes-no questions, and sentential negation.  

\subsection{Word order}

The word order modules address the so-called basic word order in
a language: the relative order of subjects, verbs, and verbal complements.
Languages vary in their rigidity with respect to word order, and the
question of how to determine the basic word-order of a language is 
notoriously complex.  Nonetheless, we believe that many if not most
linguists working on linguistic description would analyze some orders
as primary and others as derived.  Thus the basic word order modules
are meant to capture the relative ordering of subjects, verbs, and complements
in clauses which don't also involve word-order changing phenomena
such as topicalization, extraposition, subject-verb (or subject-auxiliary)
inversion, etc.  Eventual modules for such phenomena would need to
interact appropriately with the basic word-order modules.

The Matrix core grammar provides definitions of basic
head-complement and head-subject schemata which are consistent with
the Matrix implementation of compositional semantics,
%Save self-reference for full paper
%\cite{Fli:Ben:03}, 
as well as definitions of head-initial and
head-final phrase types.  These rules are also consistent with the
{\hpsg}-style handling of long-distance dependencies implemented in the
Matrix.
%Cut this footnote for space reasons
%\footnote{We note that the long-distance dependency machinery
%might be best treated as a module itself rather than part of the core
%grammar.}  
With this infrastructure, the implementation of the basic
word order modules was mostly a matter of creating subtypes joining
head-complement and head-subject schemata with the right types
specifying head/dependent order, in addition to creating instances of
those types as required by the \lkb\ parser.

The word order modules 
%implemented so far 
handle 
SOV, SVO, VSO, VOS, OVS, OSV, V-final, V-initial, and
free word order.
%Cut footnote for space reasons.
%\footnote{VOS, OVS and OSV are extremely rare among
%the world's languages, with OVS being arguably nearly unattested.
%Nonetheless, we decided to implement support for all of them.}  
In the
strict word orders, only one head-complement and one head-subject rule
are posited, with a strict ordering between them.  For V-final and
V-initial order (where the order of subjects and complements are
unconstrained), %again we have only two rules, but they are 
the two rules are allowed to
apply in either order.  For  free word order, we posit both
head-initial and head-final head-complement and head-subject rules.
This alone is enough to capture the six possible word orders, but
without further work it leads to considerable spurious ambiguity in
the cases where there are arguments on both sides of the verb.  To
solve this problem, we adapted a strategy from the English Resource
Grammar for requiring attachment to the right before attachment to the
left.\footnote{We believe that this approach is linguistically
appropriate as long as the languages in question cannot be said to
have a VP constituent in OVS clauses.  Matrix grammars produce
maximally binary-branching structures.  This is convenient in that it
reduces the number of rules a grammar needs to encode, though at a
cost of arguably more complex trees.  We maintain that not all of the
constituents represented in such a tree need to be interpreted as
linguistically meaningful.  Conversely, however, a grammar
should posit a constituent where the linguistic evidence says there
must be one.}
Figure~\ref{wordorderfig} shows the type files for the SOV and
V-final orders and the instance file they both share.

\begin{figure*}[ht]
\begin{center}
{\tt\small
\begin{tabular}{l}
\hline
{\bf SOV.tdl}\\
comp-head-phrase := basic-head-1st-comp-phrase \& head-final.\\
subj-head-phrase := basic-head-subj-phrase \& head-final \&\\
   \phantom{foo}[ HEAD-DTR.SYNSEM.LOCAL.CAT.VAL.COMPS $\langle$ $\rangle$ ].\\
{\bf V-final.tdl}\\
comp-head-phrase := basic-head-1st-comp-phrase \& head-final.\\
subj-head-phrase := basic-head-subj-phrase \& head-final.\\
{\bf V-final-rules.tdl}\\
comp-head := comp-head-phrase.\\
subj-head := subj-head-phrase.\\
\hline
\end{tabular}
}
\end{center}
\caption{SOV and V-final basic word order modules}
\label{wordorderfig}
\end{figure*}


The core Matrix also provides a lexical type hierarchy providing basic
types for, e.g., nouns and verbs which can be
cross-classified with types for argument structure patterns.
For the purposes of testing the module presented here, we made a set
of lexical types inheriting from these dimensions, instantiated by a
test lexicon.  (In future work, we plan to add a set of modules
supporting the creation of language-specific lexical types and lexical
entries instantiating those types.)  In addition, to finesse the issue
of NP structure while still producing well-formed semantic
representations (for testing via generation), we posited an
underconstrained determinerless NP rule.  This should eventually be
replaced by a set of modules representing different patterns of
optionality of determiners.

We have been speaking of these modules as handling the order of
subjects, complements, and verbs, but two caveats are in order.  First,
the head-complement and head-subject rules posited are not constrained
to any particular part of speech ({\sc head}) type. Thus, the system that is
produced models languages with consistent ordering of heads and
dependents, for heads of all parts of speech.  The Matrix provides a
type hierarchy of {\sc head} values that allows for statements over any
disjunction of head types, so it should be straightforward to produce
a system in which linearization is relativized to the head type, but
this remains future work.  Second, we began by considering clauses
with at most one complement, but of course that is an
oversimplification.  The rules as stated will handle an unbounded
number of complements, and realize them in an order determined by
their order on the lexical head's complements ({\sc comps}) list.

The next obvious step is to allow some flexibility in the order of
complements.  We believe that this will be best handled with separate,
interacting modules, as it appears that languages vary in the
extent to which they allow complements to reorder.
%\footnote{Contrast,
%for example, English, which allows PP complements to reorder with each
%other and with adjuncts fairly easily but keeps NP complements close
%to their selecting heads, with Japanese, which generally allows free
%ordering of complements and indeed free ordering between complements
%and subjects (`scrambling').}  
As a first step in this direction,
however, we provide a head-2nd-comp-phrase which realizes the
second element of a head's {\sc comps} list, while passing the first
complement up to be realized at a higher level in the tree.  This rule
is instantiated for the free-word-order module only.  For sentences
with ditransitive verbs (and thus four elements: subject, verb, direct
object, and indirect object), a grammar with the free-word order
module will license all 24 possible orders of major clause
constituents, while assigning exactly one parse to each.

Finally, we note two further aspects of this domain which will need
to be handled to produce a system useful across a broad range of
languages: (i) mixed word order, where two or more word
orders are arguably basic, but the choice between them is syntactically
constrained: either by head type (as discussed above) or by clause
type (main v.\ subordinate), and (ii) a representation of the
pragmatic effects of word order, especially in free word-order
languages.
%\footnote{In languages with relatively fixed word order, the
%displacement constructions which produce alternative orders provide a
%place to encode such pragmatic constraints.}


\subsection{Yes-no questions}

The modules for yes-no questions implement four alternative mechanisms:
(1) inversion of the subject and the main verb relative to the declarative
sentence word order; (2) inversion of the subject and an auxiliary
verb; (3) introduction of a question particle either sentence-initially or
sentence-finally; and (4) use of intonation alone to indicate a question.

Inversion of the subject and the main verb is implemented with a lexical
rule which relocates the subject (the value of {\sc subj} in the valence
specifications) to be the first on the {\sc comps} list, and further
assigns a positive value for an additional feature {\sc inv} (inverted)
on verbs.  This feature may well have independent syntactic motivation in 
the language, but is in any case used here to encode the relevant distinction
so the declarative/interrogative distinction can be made in the semantics
once the clause is constructed.

Subject-aux inversion is a minor extension of the basic inversion type,
simply constraining the lexical rule to only apply to auxiliary verbs.
This module does not deal with the special case of an additional `support'
verb like {\it do} used in English yes-no questions lacking any other
auxiliary verb, though it could be so extended if such support verbs
proved to be needed for other languages.

The third type of mechanism employs a distinct question particle, here
treated as a pre- or post-modifying sentence adverb.  The user is
prompted for this positional distinction, and for the spelling of the
particle; the code for the relevant lexical entry is then
autogenerated, instantiating a question particle type which supplies
the remaining syntactic and semantic constraints needed.

A fourth mechanism for yes-no questions is accommodated where word order is
unchanged and intonation alone distinguishes the question from the statement.
Given the underspecification of the semantics of {\it messages} (e.g.,
propositions vs. interrogatives) in the Matrix, no additional code needs to
be generated.  The user is instead advised that if intonation is marked in 
the written language, as with punctuation, further work will be needed in order
to link such a marker with the intended semantic distinction.

\subsection{Sentential negation}

The sentential negation modules handle two basic strategies, as well
as a hybrid: sentential negation through verbal inflection; a V,
VP, or S adverb (pre- or post-head); and inflection in complementary
distribution with a negative adverb. We will illustrate two of these
here.  We plan to add support for negation as a
selected adverb which appears as the complement of a (main or auxiliary) verb 
and inflection in combination with a selected adverb,
optionally or obligatorily.

The negative inflection approach to sentential negation is implemented
by means of a lexical rule, shown in Figure~\ref{negrulefig}.  This
rule takes a verbal lexeme as its daughter and produces another verbal
lexeme with negative inflection as well as negative semantics.  The
constructional content ({\sc c-cont}) of this rule adds the negative
relation and the appropriate scopal constraint.
% between that relation
%and the verb's semantics. 
In the grammar generated using this module,
this type is instantiated by a lexical rule instance, specialized
according to user input as to whether the relevant inflection is a
prefix or a affix, and as to the spelling of the suffix.  We assume
this spelling to be the underlying form which can then be passed to a
morphophonological component capable of handling non-concatenative
effects or other intricacies.  As further modules are developed, we
will need to tune the interaction of this rule with other lexical
rules to allow linguists to capture the ordering of the
processes.
%\footnote{Note, however, that the same generalizations could be
%expressed in a morphological analysis component.  In this case
%overgeneration on the part of the (morpho-)syntax would be merely
%inefficient.}

\begin{figure}[ht]
{\scriptsize
\begin{verbatim}
negation-lex-rule := infl-ltol-rule &
[ SYNSEM [ LOCAL.CAT #cat,
           NON-LOCAL #non-loc ],
  DTR.SYNSEM [ LOCAL [ CAT #cat & [ HEAD verb ],
                       CONT.HOOK [ INDEX #ind, LTOP #larg, XARG #xarg ]],
               NON-LOCAL #non-loc ],
  C-CONT [ HOOK [ INDEX #ind, LTOP #ltop, XARG #xarg ],
           RELS <! arg1-relation & [ PRED '_neg_r_rel, LBL #ltop, ARG1 #harg ] !>,
           HCONS <! qeq & [ HARG #harg, LARG #LARG ] !> ]].
\end{verbatim}
}
\caption{Inflectional rule for sentential negation}
\label{negrulefig}
\end{figure}

In the negative adverb case, we posit a lexical type which inherits
from the Matrix type for scopal adverbs.  This type is specialized
according to user input to be a pre- or post-head modifier and to
attach to S, VP, or V.  Further elaborations would allow for some
underspecification along both dimensions.  Finally, a lexical entry is
generated using an orthographic representation
%the negative adverb type so created and a spelling
provided by the user.

\subsection{Summary}

This work sheds some light on the nature of grammar engineering in the
\hpsg\ framework with respect to modularity.  What looks like a module
from a typological perspective (word order, sentential negation,
etc.)\ is in fact tightly integrated with the rest of the grammar from
the point of view of an individual language.  
%In the context of this
%project,
This means that care must be taken to ensure that the various
modules interact properly, and in fact one module sometimes needs to
explicitly say something about another part of the system.  For
example, the free word order module posits an additional feature {\sc
attach} to require rightward attachment before leftward attachment of
arguments.  The rest of the grammar needs to be modified to pass this
feature around.  For handling such cases we found ourselves making
extensive use of a purpose-built extension to TDL syntax which allows
constraints to be added to an already declared type.  At the same
time, working at a typological level has revealed 
%another kind of
%modularity: 
the potential of another kind of reuse, viz., of
%the possibility of reusing 
pieces of analysis across
different modules, for example, reusing the same rule instances for
the V-final, SOV and OSV word order modules.


\section{Management of modules}
\label{management}

There are at least three partially conflicting pressures on the design
of a method for module management in this system: 1. From a module developer's
perspective, it is desirable to capture as many generalizations as
possible across modules within a system so that common information is
factored out and coded only once, even if it is used in many
alternative modules. 2. From a grammar developer's perspective, it should be clear
what modules are available, what choices have been indicated so far,
and what implications those choices have for the generated grammar.
3. Again from a grammar developer's perspective, it should be relatively simple to
integrate updates to the Matrix (core grammar and modules) into a
grammar under development, without loss of information.
%\footnote{In
%the worst case, the grammar developer might have to manually re-tweak
%some module-generated part of the grammar for which the module had
%been updated.  However, it should always be apparent to the user when
%this is required.}  
Similarly, it should be possible for the grammar
developer to revise a decision that was made early on (such as in the
type of negation strategy the language uses) and regenerate the
automatically generated portions of the grammar, without loss of
information.  These considerations have implications both for the
encoding of the underlying knowledge base that makes up the modules
and for the user interface designed to allow a linguist to interact
with the system.

The simplest approach to these issues would be to code each module as a
separate file or files, and devise a grammar-loading script with
suitable in-line comments indicating which files to include for which
linguistic effects.  This naive approach does not do well on the first
criterion: creating all of those separate files represents a great
deal of redundancy (sentential negation with preverbal VP adverbs is
very much like sentential negation with postverbal S adverbs).
%, in terms
%of the code that needs to be produced).  
%We could address this by
%automatically generating the various files while still presenting them
%to the user as static entities.  However, 
In addition, this won't scale up to
situations where the exact contents of one module need to be
customized in light of other modules chosen.  The naive approach fails
rather miserably on the second, UI-focused, criterion.  It 
%would be
%desirable 
preferable to present the user with linguistic questions in a dynamic
fashion (presenting the choice of type of negative adverb only after
that general strategy has been chosen, for example).  This approach
does answer consideration (3) fairly well, however
%: it would be simple enough to
%provide a tool which compares the user's versions of the files
%included to off-the-shelf copies of the same modules, and alerts the
%user to any local changes to files which have either been updated by
%the Matrix developers or are involved in a grammar revision (change
%of module selection).  However, 
even here there is still room for improvement:
some of the customization (e.g., spellings of various morphemes) could
be automatically updated more easily if that information were stored
where the system had direct access to it, rather than implicitly in
modified files.

A better solution would draw on linguistic knowledge stored in
separate files to automatically produce both the files required by a
particular grammar and the script that loads them.  The UI 
%for this
%system 
would be a form or wizard which allows user input of both
broad choices (e.g., which word order system to use) and information
to customize modules (Is the negative inflection a prefix or a suffix?
How is it spelled?).  Such subsidiary questions would only appear once
the user picks an option for which they are relevant. 
%Furthermore, the
%system would record the user's answers, and perhaps even be able to
%take a grammar and work out which answers must have been given, as
%well as where files have been modified.
Furthermore, the system would record the user's answers 
and be able to detect where files have been modified, in order to
support updates to the Matrix/modules or revisions to the user's choices.

In this paper, we leave the development of such a UI for future work,
and develop instead a Perl script that steps the user through the
linguistic questions (raising only those that become relevant) and
generates both the grammar files and the grammar-loading script which
the \lkb\ uses to call them.  While the command-line UI is certainly far
from ideal, it does succeed in simplifying the user's task of
customizing the Matrix.
% by making selections along the three dimensions
%discussed in \S\ref{imp}.  
For example, instead of writing the TDL code or reading and
selecting among the various types and instances files for the word
order modules, the user need only answer the multiple choice question
``What is the basic word order of your language?'' and the appropriate
rules and instances are incorporated in the generated grammar.

%At the moment, the modules produce
%several separate files, though it is an open question whether
%it would be easier for users to further modify such automatically
%generated grammars if the grammar code were more consolidated.

\section{Validation of prototype system}

To verify the mutual consistency of the modules developed so far as
well as to illustrate their applicability to a interesting range of
languages, we developed abstract test suites for seven languages, in
each case including both grammatical and ungrammatical examples.  This
sample of languages is not intended to be representative, either
typologically or genetically; rather it is a convenience sample.  The
test suites are abstract in that we represent sentences in terms of
strings of part of speech tags rather than particular words, in order
to sidestep the problems of lexica and morphology for each language.
The languages and their properties are shown in
Table~\ref{testsuitetable}.

\begin{table*}[ht]
\begin{center}
\small
\begin{tabular}{llll}
\hline
Language\footnotemark & Word order & Negation & Yes-no Q\footnotemark \\ \hline
Hindi    & SOV        & pre-V adv 	& sentence-initial particle\\
Japanese & V-final    & verbal suffix   & sentence-final particle\\
Mandarin & SVO	      & pre-V adv	& sentence-final particle, A-not-A \\
Polish   & free	      & pre-V adv       & sentence-initial particle  \\
Slave (Athabaskan) & SOV & post-V adv	& sentence-initial particle \\
English  & SVO	      & post-aux adv	& aux inversion \\
Spanish  & SVO	      & pre-V adv	& main verb inversion \\ \hline
\end{tabular}
\end{center}
\caption{Languages used in testing}
\label{testsuitetable}
\end{table*}

\addtocounter{footnote}{-1}
\footnotetext{Sources: Hindi: \citeboth{Sne:Wei:00},
Mandarin: \citeboth{Li*81}, Polish: Adam Przepi\'{o}rkowski, p.c.,
Slave: \citeboth{Rice89}}
\addtocounter{footnote}{1}
\footnotetext{In addition to intonation questions, if any.}
%\addtocounter{footnote}{1}
%\footnotetext{The situation in English is a bit more complicated.  \namecite{Kim00} argues that sentential {\it not} is actually selected by the auxiliary it appears with.}

We ran the Perl script once for each language, answering the
multiple-choice questions with single key strokes; fewer than ten
keystrokes were required for any one grammar, excluding input for
spelling.  The script then automatically generated the
language-specific grammar files and the customized load file
which we used to load the new grammar.  We then batch parsed the small
test suite for that language, and repeated the exercise for the next
language.  Table~\ref{testresults} presents the coverage of the
automatically generated grammars for the seven languages.

\begin{table*}[ht]
\begin{center}
\small
\begin{tabular}{lrrrr}
\hline
Language & Positives & Coverage & Negatives & Overgeneration \\ \hline
Hindi    & 5         & 100\% 	& 10 & 10\% \\
Japanese & 6         & 100\%    &  8 & 0\% \\
Mandarin & 4	     &  75\%	&  9 & 0\% \\
Polish   & 14	     & 100\%    &  8 & 0\% \\
Slave    & 3         & 100\%	&  6 & 0\% \\
English  & 5	     & 80\%	& 11 & 45\%\\
Spanish  & 5         & 80\%	& 8  & 25\% \\ \hline
\end{tabular}
\end{center}
\caption{Parsing evaluation results}
\label{testresults}
\end{table*}

Clearly, the resulting grammars are not all perfect.  For Mandarin,
there is as yet no provision for A-not-A questions.  For
English, the negated yes-no question example fails to
parse, since we falsely treated the negation particle as a modifier
rather than a complement (cf.\ \citeboth{Kim00}),
which also leads to considerable overgeneration. For Spanish, the
test suite included declarative/interrogative word order contrasts, but
without a treatment of punctuation, the grammar confuses one for the other.
Otherwise, these grammars perform quite well on these modest data sets,
and reflect an almost trivial implementation effort on the part of the user,
contrasted with a day or more of skilled grammar development work to 
achieve similar results with the standard Matrix.

\section{Conclusion and outlook}

We have described a method for extending a language-independent core
grammar like the Grammar Matrix with modules handling
cross-linguistically variable but still recurring patterns.  This
method allows for extremely rapid prototyping of deep precision
grammars in such a way that the prototypes themselves can serve as the
basis for sustained development.  We envision at least five potential
uses for this kind of grammar prototyping: (i) in pedagogical
contexts, where it would allow grammar engineering students to more
quickly work on cutting-edge problems, (ii) in language documentation,
where a documentary linguist in the field might be collaborating
remotely with a grammar engineer to propose and test hypotheses, (iii)
in leveraging the results from economically powerful languages to
reduce the cost of creating resources for minority languages, (iv) in
presenting typological properties in machine-readable format to
improve ML algorithms which might need to be tuned differently for
different kinds of languages, and (v) in supporting typological or
comparative studies of linguistic phenomena or interactions between
phenomena across languages.

%\section*{References}

%\bibliographystyle{acl}
\bibliographystyle{robbib}
\bibliography{modules}


\end{document}

